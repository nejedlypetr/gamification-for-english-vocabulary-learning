\begin{abstract-czech}
    Tato práce se zabývá integrací gamifikačních prvků do mobilní aplikace English Mind zaměřenou na výuku anglických slovíček s cílem zvýšit zapojení uživatelů a zlepšit výsledky výuky. Prostřednictvím analýzy stávajících metodik a konkurenčních aplikací byly identifikovány a implementovány čtyři klíčové prvky gamifikace: rozmanité typy kartiček, sledování pokroku jednotlivých slovíček, analýza po skončení procvičování a streak systém.

    Implementace zahrnovala restrukturalizaci architektury aplikace s využitím monorepo přístupu pomocí systému Melos, což umožnilo paralelní vývoj hlavní aplikace a demonstračního prototypu. Uživatelské testování ukázalo vysokou míru spokojenosti a potvrdilo účinnost implementovaných funkcí.

    Výsledky ukazují, že pečlivě navržené gamifikační prvky mohou výrazně zlepšit konkurenční pozici aplikace pro výuku jazyků s ohledem na zachování jejích základních výukových metodik.
\end{abstract-czech}
    
\begin{abstract-english}
    This thesis explores the integration of gamification elements into English Mind, a language learning mobile application, to enhance user engagement and learning outcomes. Through analysis of existing methodologies and competitor applications, four key gamification features were identified and implemented: diversified flashcard types, vocabulary progress tracking, post-practice analytics, and a streak system.

    The implementation involved restructuring the system architecture using a monorepo approach with Melos, enabling parallel development of the main application and a demonstration prototype. User testing showed strong satisfaction, confirming the effectiveness of the implemented features.

    The results demonstrate that carefully designed gamification elements can significantly improve a language learning application's competitive position while maintaining its core learning methodologies.
\end{abstract-english}