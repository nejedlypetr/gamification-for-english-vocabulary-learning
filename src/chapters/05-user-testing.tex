\chapter{User Testing}

This chapter describes the methodology and results of user testing conducted on the previously proposed and designed gamification features for English Mind. The testing focused on evaluating the clarity, intuitiveness, and potential effectiveness of the new gamification elements.

\section{Testing Methodology}

Two distinct user groups were selected for testing to provide diverse perspectives. Group A consisted of participants with prior experience using language learning applications, while Group B included participants without significant experience in this area.

\begin{itemize}
    \item \textbf{Group A (Experienced Users):}
    \begin{itemize}
        \item 6 participants aged 18-45
        \item Regular users of apps like Duolingo, WordUp, or Duocards
        \item Familiar with gamification concepts through regular usage of language learning applications (even without knowing the formal terminology)
    \end{itemize}

    \item \textbf{Group B (Novice Users):}
    \begin{itemize}
        \item 6 participants aged 18-45
        \item Interest in learning English vocabulary
        \item Limited exposure to language learning applications
    \end{itemize}
\end{itemize}

The testing was conducted in a controlled, quiet environment where participants interacted with high-fidelity interactive prototypes on mobile devices. Each session lasted between 30-45 minutes and was documented through screen and audio recordings.

Data collection during the testing sessions combined several methods to gather comprehensive feedback about the gamification features. Participants were encouraged to think aloud while interacting with the prototype, sharing their thoughts and reactions to different elements. This verbal feedback was supplemented by observation notes documenting user behavior, particularly noting any points of confusion or excitement. After completing the test scenarios, participants filled out a structured questionnaire rating various aspects of the gamification features on a 5-point Likert scale (see Table \ref{tab:questionnaire}). The questionnaire focused on key metrics such as feature clarity, perceived usefulness, and likelihood of maintaining engagement.


\begin{table}[h]
    \centering
    \caption{User Testing Questionnaire}
    \label{tab:questionnaire}
    \makebox[\textwidth][c]{
        \begin{tabular}{|p{0.9\textwidth}|c|c|c|c|c|}
            \hline
            \multicolumn{6}{|l|}{\small 1 = Strongly Disagree, 2 = Disagree, 3 = Neutral, 4 = Agree, 5 = Strongly Agree} \\
            \hline
            \textbf{Question} & \textbf{1} & \textbf{2} & \textbf{3} & \textbf{4} & \textbf{5} \\
            \hline
            \multicolumn{6}{|l|}{\textbf{Practice Flashcards Experience}} \\
            \hline
            1. The different types of flashcards made practice more engaging & & & & & \\
            \hline
            2. Each flashcard type's instructions were clear and easy to understand & & & & & \\
            \hline
            3. The variety in flashcard types helped me learn vocabulary more effectively & & & & & \\
            \hline
            4. The distribution of different flashcard types felt well-balanced & & & & & \\
            \hline
            5. The five-stage individual word progress indicator was easy to understand & & & & & \\
            \hline
            6. Seeing my progress for individual words motivated me to practice more & & & & & \\
            \hline
            7. The individual word progress indicator's placement was visually clear without being distracting & & & & & \\
            \hline
            8. I found the progress tracking helpful for understanding my learning journey & & & & & \\
            \hline
            9. The practice statistics provided useful information about my session & & & & & \\
            \hline
            10. The celebratory animations made completing practice more rewarding & & & & & \\
            \hline
            11. The review screen motivated me to complete future practice sessions & & & & & \\
            \hline
            \multicolumn{6}{|l|}{\textbf{Streak System}} \\
            \hline
            12. The streak system's requirements were clear and easy to understand & & & & & \\
            \hline
            13. The visual states (active/at risk) effectively communicated streak status & & & & & \\
            \hline
            14. The celebration screen for maintaining streaks felt rewarding & & & & & \\
            \hline
            15. The requirement to learn one new word daily feels achievable & & & & & \\
            \hline
            16. The streak system would help me build a consistent practice habit & & & & & \\
            \hline
            17. I would be more likely to use the app regularly because of the streak feature & & & & & \\
            \hline
            \multicolumn{6}{|l|}{\textbf{Overall Experience}} \\
            \hline
            18. The gamification features enhanced my learning experience & & & & & \\
            \hline
            19. The features felt well-integrated with the app's educational purpose & & & & & \\
            \hline
            20. The gamification elements maintained my interest without being distracting & & & & & \\
            \hline
            21. I would recommend this app to others learning English vocabulary & & & & & \\
            \hline
            22. I would continue using this app for long-term vocabulary learning & & & & & \\
            \hline
        \end{tabular}
    }
\end{table}

\section{Test Scenarios}

\subsection*{Test Scenario 1: Practice Flashcards Experience}
This scenario evaluates the core practice flow, focusing on three key gamification elements that users encounter during a typical practice session:

\begin{enumerate}
    \item \textbf{Various Flashcard Types}
    \begin{itemize}
        \item Clarity of instructions for each flashcard type 
        \item Perceived value of variety in maintaining engagement
        \item Comfort level with each flashcard type's interaction method
    \end{itemize}

    \item \textbf{Individual Word Progress Tracking}
    \begin{itemize}
        \item Recognition and interpretation of the five-stage progress indicator
        \item Understanding how progress advances through practice
        \item Visibility and placement of the progress indicator
        \item Motivational impact of seeing word progress
    \end{itemize}

    \item \textbf{Post-Practice Review}
    \begin{itemize}
        \item Comprehension of practice statistics
        \item Impact of celebratory animations and mascot interactions
    \end{itemize}
\end{enumerate}

Participants will complete a full practice session, encountering each element naturally. They will be encouraged to share their thoughts about the gamification features as they experience them, with particular attention to: intuitiveness of the interface, emotional response to progress indicators and celebrations, perceived effectiveness in maintaining motivation, and any points of confusion or suggestions for improvement.

\subsection*{Test Scenario 2: Streak System}

This scenario evaluates the user's understanding and perception of the streak system, focusing on how users interact with and feel about maintaining streaks:

\begin{enumerate}
    \item \textbf{Understanding Streak Activation and Maintenance}
    \begin{itemize}
        \item Clarity of instructions on how to activate and maintain a streak
        \item Awareness of daily requirements to keep the streak active
        \item Recognition of streak status indicators and their meanings
    \end{itemize}

    \item \textbf{Motivational Impact}
    \begin{itemize}
        \item Perceived motivation to engage with the app regularly due to the streak system
        \item Feedback on the visual and interactive elements associated with streaks
    \end{itemize}
\end{enumerate}

Participants will be guided through a scenario that simulates maintaining a streak and will be encouraged to share their thoughts on the streak system. The focus will be on their understanding of the requirements for maintaining a streak, their emotional engagement, and any suggestions for improving the streak experience.

\section{Results and Evaluation}

TODO...
