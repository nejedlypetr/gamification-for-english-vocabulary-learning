\chapter{User Testing}

This chapter describes the methodology and results of user testing conducted on the previously proposed and designed gamification features for English Mind. The testing focused on evaluating the clarity, intuitiveness, and potential effectiveness of the new gamification elements.

\section{Testing Methodology}

Two distinct user groups were selected for testing to provide diverse perspectives. Group A consisted of participants with prior experience using language learning applications, while Group B included participants without significant experience in this area.

\begin{itemize}
    \item \textbf{Group A (Experienced Users):}
    \begin{itemize}
        \item 6 participants aged 18-45
        \item Regular users of apps like Duolingo, WordUp, or Duocards
        \item Familiar with gamification concepts through regular usage of language learning applications (even without knowing the formal terminology)
    \end{itemize}

    \item \textbf{Group B (Novice Users):}
    \begin{itemize}
        \item 6 participants aged 18-45
        \item Interest in learning English vocabulary
        \item Limited exposure to language learning applications
    \end{itemize}
\end{itemize}

The testing was conducted in a controlled, quiet environment where participants interacted with high-fidelity interactive prototypes on mobile devices. Each session lasted between 30-45 minutes and was documented through screen and audio recordings.

Data collection during the testing sessions combined several methods to gather comprehensive feedback about the gamification features. Participants were encouraged to think aloud while interacting with the prototype, sharing their thoughts and reactions to different elements. This verbal feedback was supplemented by observation notes documenting user behavior, particularly noting any points of confusion or excitement. After completing the test scenario (see Section \ref{sec:test-scenario}), participants filled out a structured questionnaire rating various aspects of the gamification features on a 5-point Likert scale (see Table \ref{tab:questionnaire}). The questionnaire focused on key metrics such as feature clarity, perceived usefulness, and likelihood of maintaining engagement.

\input{src/chapters/questionnaire}

\section{Test Scenario}
\label{sec:test-scenario}
The test scenario consists of two main phases: understanding the streak system and completing a practice session. This approach allows evaluation of both the gamification mechanics and their integration into the learning experience:

\begin{enumerate}
    \item \textbf{Phase 1: Streak System}
    \begin{itemize}
        \item Understanding streak activation and maintenance requirements
        \item Interpreting streak status indicators
        \item Initial feedback on motivational aspects
    \end{itemize}
    
    \item \textbf{Phase 2: Practice Flashcards Session}
    \begin{itemize}
        \item {Various Flashcard Types}
        \begin{itemize}
            \item Clarity of instructions for each flashcard type 
            \item Perceived value of variety in maintaining engagement
            \item Comfort level with each flashcard type's interaction method
        \end{itemize}

        \item {Individual Word Progress Tracking}
        \begin{itemize}
            \item Recognition and interpretation of the five-stage progress indicator
            \item Visibility and placement of the progress indicator
            \item Motivational impact of seeing word progress
        \end{itemize}

        \item {Post-Practice Review}
        \begin{itemize}
            \item Comprehension of practice statistics
            \item Impact of celebratory animations and mascot interactions
            \item Understanding of how the completed session affects their streak
        \end{itemize}
    \end{itemize}
\end{enumerate}

The testing process employs a think-aloud protocol, where participants verbalize their thoughts while interacting with the prototype. Initially, participants demonstrate their understanding of the streak system to the interviewer. They then proceed through a complete practice session, experiencing the integrated gamification elements. Throughout both phases, participants provide continuous feedback on interface clarity, emotional engagement, perceived long-term motivation potential, and any usability concerns. This structured approach enables comprehensive evaluation of both immediate user experience and potential sustained engagement with the application.

\newpage

\section{Results and Evaluation}

TODO...