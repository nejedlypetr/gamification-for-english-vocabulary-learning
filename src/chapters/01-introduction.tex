\chapter{Introduction}

In today's fast-paced digital world, language learning applications have become an essential tool for individuals seeking to enhance their vocabulary and language skills. However, maintaining user engagement and motivation over time remains a significant challenge. This project addresses this issue by exploring the integration of gamification elements into English Mind, a language learning mobile application, aiming to transform the learning experience into a more engaging and rewarding journey.

English Mind is built on a well-designed learning approach that effectively supports language acquisition. However, when compared to leading competitors, it falls short in terms of gamification elements. This lack of engaging features may negatively impact user motivation and retention, as users are increasingly drawn to applications that offer interactive and rewarding experiences. 

This project aims to address this gap by integrating gamification strategies that can enhance user engagement and motivation, thereby positioning English Mind more competitively in the market.

\newpage

\section{Success Criteria}
\label{sec:success-criteria}

The evaluation of this work's success is structured around three principal dimensions: competitive positioning, user feedback, and implementation success. These dimensions serve as the foundation for assessing the achievement of the thesis objectives.

\begin{enumerate}
    \item \textbf{Enhancement of Competitive Positioning}
    \begin{enumerate}
        \item Analyze the English Mind vocabulary learning approach to understand the key principles of the application.
        \item Conduct analysis of gamification features used by English Mind and leading competitors.
        \item Identify opportunities for enhancing the gamification of English Mind.
    \end{enumerate}

    \item \textbf{Positive User Feedback on Gamification Features}
    \begin{enumerate}
        \item Design gamification features and elements that are on par with or surpass those of leading competitors.
        \item Conduct user testing of the new gamification features, aiming for positive feedback.
    \end{enumerate}

    \item \textbf{Successful Implementation of Gamification Features}
    \begin{enumerate}
        \item Implement the newly designed gamification features.
        \item Conduct testing with end users to ensure functionality and user satisfaction.
    \end{enumerate}
\end{enumerate}
