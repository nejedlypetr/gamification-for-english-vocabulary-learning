\chapter{Introduction}

In today's fast-paced digital world, language learning applications have become an essential tool for individuals seeking to enhance their vocabulary and language skills. However, maintaining user engagement and motivation over time remains a significant challenge \cite{cite:govender2021_gamification_elements_in_language_learning_apps}. This project addresses this issue by exploring the integration of gamification elements into English Mind \cite{cite:english_mind_website}, a language learning mobile application. Aiming to transform the learning experience into a more engaging and rewarding journey.

English Mind, the subject of this study, is built on a well-designed learning approach that effectively supports language acquisition. However, when compared to leading competitors \cite{cite:duolingo, cite:wordup, cite:duocards}, it falls short in terms of gamification elements. This lack of engaging features may negatively impact user motivation and retention, as users are increasingly drawn to applications that offer interactive and rewarding experiences \cite{cite:deterding2011_gamefulness, cite:govender2021_gamification_elements_in_language_learning_apps}. 

This project aims to enhance user engagement and motivation by carefully selecting and implementing gamification features that align with English Mind's core pedagogical principles. As the application's developer, this research benefits from an in-depth understanding of its architecture and pedagogical foundations, ensuring that any added gamification elements complement rather than compromise the existing learning approach.

This thesis is organized into eight chapters. After introducing the motivation and success criteria (Chapter~\ref{sec:success-criteria}), the current state and learning approach of English Mind are analyzed (Chapter~\ref{chap:mobile-application-english-mind}). Gamification principles in leading applications are then examined (Chapter~\ref{chap:gamification-analysis}), followed by the selection of features for implementation that align with English Mind's core pedagogical principles (Chapter~\ref{chap:proposed-solution-design}). The selected features are validated through user testing of an interactive prototype (Chapter~\ref{chap:prototype-user-testing}), implemented (Chapter~\ref{chap:implementation}), and thoroughly tested by end users (Chapter~\ref{chap:testing}). Finally, the findings are summarized, success criteria are evaluated, and future improvements are suggested (Chapter~\ref{chap:conclusion}).

\newpage

\section{Success Criteria}
\label{sec:success-criteria}

The main goal of this work is to enhance English Mind by adding gamification features that align with its core pedagogical principles while maintaining the effectiveness of its learning approach. Based on this goal, the evaluation of this work's success is structured around three principal dimensions: competitive positioning, user feedback, and implementation success.

\begin{enumerate}
    \item \textbf{Enhancement of Competitive Positioning}
    \begin{enumerate}
        \item Analyze the English Mind vocabulary learning approach to understand the key principles of the application.
        \item Conduct analysis of gamification features used by English Mind and leading competitors.
        \item Identify opportunities for enhancing the gamification of English Mind.
    \end{enumerate}

    \item \textbf{Positive User Feedback on Gamification Features}
    \begin{enumerate}
        \item Design gamification features and elements that are on par with or surpass those of leading competitors and align with English Mind's core pedagogical principles.
        \item Conduct user testing of an interactive prototype showcasing the new gamification features, aiming for positive feedback.
    \end{enumerate}

    \item \textbf{Successful Implementation of Gamification Features}
    \begin{enumerate}
        \item Implement the newly designed gamification features while preserving the application's existing learning approach.
        \item Conduct a long-term testing with end users to ensure functionality and user satisfaction.
    \end{enumerate}
\end{enumerate}
