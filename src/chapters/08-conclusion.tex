\chapter{Conclusion}

This thesis has explored the integration of gamification elements into English Mind, a language learning application, with the aim of enhancing user engagement and learning outcomes. Through careful analysis of existing methodologies and competitor applications, key opportunities for improvement were identified and a set of carefully designed gamification features were implemented. 

The success of this work is evaluated against the three principal dimensions defined in the Introduction \ref{sec:success-criteria}: competitive positioning, user feedback, and implementation success.

\begin{enumerate}
    \item \textbf{Enhancement of Competitive Positioning}
    \begin{enumerate}
        \item The project successfully analyzed English Mind's core methodologies, including frequency-based vocabulary acquisition and spaced repetition, to understand the key principles of the application, see Chapter \ref{chap:mobile-application-english-mind}.
        
        \item A comprehensive analysis of three similar language learning applications was conducted, identifying several opportunities to improve the user experience and competitive positioning, see Chapter \ref{chap:gamification-analysis}.
        
        \item Four key gamification features and enhancements were identified: diversified flashcard types, vocabulary progress tracking, post-practice analytics, and a streak system, all carefully selected to align with the application's core methodologies while addressing the competitive gap, see Chapter \ref{chap:proposed-solution-design}.
        
        The comparison of gamification features in Table \ref{tab:gamification-comparison-result} shows how these enhancements filled the identified gaps in English Mind's feature set compared to its competitors.
        \newpage
        \begin{table}[h]
    \caption{Enhanced Competitive Positioning in Gamification Features (blue check mark "\textcolor{ctubluetext}{\ding{51}}"  indicates a newly added feature)}
    \label{tab:gamification-comparison-result}
    
    % Spacing between rows and columns
    \renewcommand{\arraystretch}{1.2}
    \setlength{\tabcolsep}{2pt}
    
    \begin{tabular}{l>{\centering}p{2cm}>{\centering}p{2cm}>{\centering}p{2cm}>{\centering\arraybackslash}p{2cm}}
        \toprule
        \textbf{Gamification} & \textbf{English Mind} & \textbf{WordUp} & \textbf{DuoCards} & \textbf{Duolingo} \\
        \midrule
        \multicolumn{5}{l}{\textbf{Flashcard Types}} \\
        Exercise variety & \textcolor{ctubluetext}{\ding{51}} & \ding{51} & \ding{51} & \ding{51} \\
        Recall meaning & \ding{51} & \ding{51} & \ding{51} & \ding{51} \\
        Speech recognition & \textcolor{ctubluetext}{\ding{51}} & \textemdash & \textemdash & \ding{51} \\
        Spelling assessment & \textcolor{ctubluetext}{\ding{51}} & \ding{51} & \ding{51} & \ding{51} \\
        Matching definitions & \textcolor{ctubluetext}{\ding{51}} & \ding{51} & \textemdash & \textemdash \\
        Matching translations & \textcolor{ctubluetext}{\ding{51}} & \textemdash & \ding{51} & \ding{51} \\
        \midrule
        \multicolumn{5}{l}{\textbf{Progress Tracking}} \\
        Aggregate progress metrics & \ding{51} & \ding{51} & \ding{51} & \ding{51} \\
        Practice session progress & \ding{51} & \textemdash & \textemdash & \ding{51} \\
        Word-level analytics & \textcolor{ctubluetext}{\ding{51}} & \ding{51} & \textemdash & \textemdash \\
        Post-practice analytics & \textcolor{ctubluetext}{\ding{51}} & \textemdash & \textemdash & \ding{51} \\
        \midrule
        \multicolumn{5}{l}{\textbf{Engagement Features}} \\
        Streak system & \textcolor{ctubluetext}{\ding{51}} & \textemdash & \ding{51} & \ding{51} \\
        Time-based goals & \textemdash & \ding{51} & \textemdash & \textemdash \\
        Social comparison & \textemdash & \ding{51} & \textemdash & \ding{51} \\
        \bottomrule
    \end{tabular}
\end{table}
    \end{enumerate}

    \item \textbf{Positive User Feedback on Gamification Features}
    \begin{enumerate}
        \item The four key gamification features were designed and carefully crafted to be on par with or surpass those of leading competitors while maintaining English Mind's unique value proposition, see Chapter \ref{chap:proposed-solution-design}.

        \item A prototype user testing was conducted with six participants, who provided detailed feedback through post-test questionnaires. The results showed strong user satisfaction with an average rating of 4.5/5.0 (see Table \ref{tab:questionnaire-results}), confirming the value of the proposed features while highlighting areas for improvement, see Chapter \ref{chap:prototype-user-testing}.
    \end{enumerate}

    \item \textbf{Successful Implementation of Gamification Features}
    \begin{enumerate}
        \item The current system architecture was thoroughly analyzed, revealing limitations in component coupling and documentation, see Chapter \ref{sec:existing-system-architecture}. A new monorepo solution using Melos was proposed and implemented, enabling parallel development of the main application and the demonstration prototype while preserving intellectual property, see Chapter \ref{sec:new-system-architecture}. All proposed gamification features were successfully implemented in the demonstration application, showcasing the effectiveness of the new architecture, see Chapter \ref{sec:features-implementation}. The final implemented UI is shown in Appendix \ref{app:implementation}.

        \item The implementation was validated through comprehensive end-user testing, which confirmed the robustness and usability of the new features. The testing process verified the successful integration of the gamification elements and their alignment with the application's core learning methodologies, see Chapter \ref{chap:testing}.
    \end{enumerate}
\end{enumerate}

In conclusion, this thesis has demonstrated the successful integration of gamification elements into English Mind, significantly enhancing its competitive position in the language learning application market. The implemented features not only address the identified gaps in the application's feature set but also maintain the core learning methodologies that make English Mind unique. The positive user feedback and successful implementation validate the effectiveness of the proposed solution, suggesting that gamification can indeed serve as a powerful tool for enhancing user engagement and learning outcomes in language learning applications.

Looking ahead, several promising directions for future work emerge. First, the gamification features could be expanded to include social elements, such as collaborative learning challenges or competitive leaderboards, which could further enhance user motivation and engagement. Second, a longitudinal study could be conducted to assess the long-term impact of these gamification features on vocabulary retention and learning outcomes. These future developments would not only enhance the application's capabilities but also contribute to the broader understanding of gamification's role in language learning.
