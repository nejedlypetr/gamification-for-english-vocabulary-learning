\chapter{Implementation}
\label{chap:implementation}

This chapter presents the technical implementation of the gamification features proposed in Chapter \ref{chap:proposed-solution-design} and validated through user testing in Chapter \ref{chap:prototype-user-testing}. The implementation process was structured to ensure both technical feasibility and alignment with the application's existing architecture while maintaining the core pedagogical principles of English Mind.

This chapter begins with an analysis of the current system architecture, providing context for the technical decisions made during implementation. Following this, a monorepo architecture is introduced that enables parallel development of both the main application and a demonstration application, allowing for the protection of the main application's intellectual property while still enabling development and testing of new features. This approach allows for rapid iteration and testing of new features without disrupting the production environment. Finally, the implementation of selected gamification features is detailed, focusing on technical considerations, integration challenges, and solutions adopted.

The complete source code of the demonstration application is available in a public GitHub repository \cite{cite:source_code_repository}.

\section{Existing System Architecture}
\label{sec:existing-system-architecture}

English Mind is built using the Flutter framework \cite{cite:flutter_framework}, chosen for its cross-platform capabilities and robust widget system. The application follows a modular architecture inspired by Clean Architecture principles \cite{cite:clean_architecture}, ensuring separation of concerns and maintainability. This section provides a comprehensive overview of the current system architecture, detailing the technology stack, architectural patterns, and key system components.
\newpage
\subsection{Technology Stack}

The Flutter framework enables the development of a single codebase that runs seamlessly on both iOS and Android platforms while maintaining native-like performance and user experience \cite{cite:flutter_framework}. The decision to use Flutter was driven by its ability to deliver consistent user interfaces across platforms while maintaining high performance, as demonstrated by its successful implementation in numerous production applications \cite{cite:google_flutter_case_studies}.

The data management strategy combines cloud-based and local storage solutions to balance performance, offline capability, and real-time synchronization requirements. Firebase Firestore \cite{cite:firebase_firestore} serves as the primary cloud storage solution, providing real-time data synchronization and a NoSQL document database. This choice was made to leverage Firestore's built-in offline persistence and real-time updates, which are crucial for maintaining user progress across devices. For translation services, the application integrates with the Microsoft Translator API \cite{cite:microsoft_translator}, selected for its high accuracy and extensive language support. The vocabulary data is stored locally in a JSON file containing word definitions and sentence examples, a design decision that ensures immediate access to core content without network dependency and reduces server load.

The application leverages several key Flutter/Dart packages to enhance functionality and maintainability:
\begin{itemize}
    \item \textbf{get\_it} - Dependency injection framework for managing object creation and lifetime.
    \item \textbf{flutter\_hooks} - State management solution chosen for its simplicity and performance benefits over traditional state management approaches.
    \item \textbf{auto\_route} - Navigation and routing system that provides type-safe routing and reduces boilerplate code.
    \item \textbf{objectbox} - Local storage solution for efficient data management.
    \item \textbf{sign\_in\_with\_apple} and \textbf{google\_sign\_in} - Third-party authentication providers.
    \item \textbf{firebase\_auth} - User authentication service providing secure user management.
    \item \textbf{firebase\_functions} - Cloud functions for serverless computing, enabling scalable backend operations.
    \item \textbf{firebase\_firestore} - Cloud storage for real-time data synchronization and offline persistence.
    \item \textbf{flutter\_localizations} - Localization framework enabling internationalization support.
\end{itemize}

\subsection{Architectural Overview}
\label{sec:architectural-overview}

The application follows Clean Architecture \cite{cite:clean_architecture} principles with a clear separation of concerns across distinct layers, while incorporating custom architectural decisions to better suit the application's specific needs. The architecture is designed to maintain high maintainability and scalability while ensuring a unidirectional data flow. The codebase is organized into the following directory structure, each serving a specific purpose in the application's architecture:

\begin{itemize}
    \item \textbf{\texttt{core/}} – Provides shared utilities and common functionality.
    \begin{itemize}
        \item \textbf{\texttt{constants/}} – Application-wide constants and shared values.
        \item \textbf{\texttt{extensions/}} – Dart language extensions providing additional functionality.
        \item \textbf{\texttt{injector/}} – Dependency injection configuration and service registration.
        \item \textbf{\texttt{l10n/}} – Internationalization and localization support.
        \item \textbf{\texttt{logger/}} – Application logging functionality and utilities.
        \item \textbf{\texttt{routes/}} – Application navigation and routing configuration.
        \item \textbf{\texttt{utils/}} – Utility functions and helpers.
    \end{itemize}

    \item \textbf{\texttt{domain/}} – Contains the core business logic and is implementation independent.
    \begin{itemize}
        \item \textbf{\texttt{entities/}} – Core business objects representing the domain model.
        \item \textbf{\texttt{managers/}} – Business logic orchestrators and state managers.
        \item \textbf{\texttt{repositories/}} – Abstract interfaces defining data operations.
        \item \textbf{\texttt{usecases/}} – Core business operations and rules between entities and repositories.
    \end{itemize}
    
    \item \textbf{\texttt{data/}} – Handles data operations and implementations of repository interfaces.
    \begin{itemize}
        \item \textbf{\texttt{models/}} – Data models and transformations.
        \item \textbf{\texttt{repositories/}} – Concrete implementations of repository interfaces.
        \item \textbf{\texttt{mappers/}} – Transformation logic between domain entities and data models.
    \end{itemize}
    
    \item \textbf{\texttt{ui/}} – Presentation layer managing user interface components and interactions.
    \begin{itemize}
        \item \textbf{\texttt{theme/}} – UI theme configuration and styling.
        \item \textbf{\texttt{screens/}} – Screen-specific UI components and layouts.
        \item \textbf{\texttt{widgets/}} – Reusable UI components.
        \item \textbf{\texttt{hooks/}} – Hooks for state management and resource handling within Flutter widgets.
    \end{itemize}
\end{itemize}

\subsection{Architecture Limitations}
\label{sec:architecture-limitations}

While the current architecture provides a solid foundation for the core English Mind application, it presents several challenges when implementing new gamification features. A critical consideration is the proprietary nature of the English Mind application, which necessitates a separate demonstration application (\texttt{em\_demo}) for implementing gamification features. This approach preserves intellectual property rights while providing a controlled environment for feature validation. The demonstration application serves as a proof of concept, enabling the implementation and testing of gamification elements without exposing the original application's proprietary codebase. This separation allows for independent development and validation of new features while maintaining the integrity of the production application.

The architectural challenges manifest in several key areas that impact the development of the demonstration application. First, while the current architecture follows clean architecture principles, the UI components exhibit tight coupling with application-specific concerns. This coupling creates significant challenges when attempting to reuse components in the new application.

A primary example of this coupling is the integration of multiple concerns within individual components. Specifically, UI components combine both localization logic through \texttt{flutter\_localizations} and state management through \texttt{flutter\_hooks}. This integration violates the single responsibility principle \cite{cite:solid_principles}, as each component handles multiple distinct responsibilities. The resulting tight coupling makes it challenging to extract and reuse components in the new application without also migrating their dependencies, leading to potential code duplication and increased maintenance overhead.

Additionally, the lack of systematic component documentation presents a significant barrier to component reuse. Without clear documentation of component interfaces, dependencies, and usage patterns, developers face difficulties in understanding and adapting existing UI elements for the demonstration application. This issue is particularly problematic when attempting to maintain consistency between the original and demonstration applications.

These limitations collectively highlight the need for a more modular architecture that supports component isolation and documentation. Such an architecture would enable better component reuse while maintaining consistency between the original and demonstration applications.
\newpage
\section{New System Architecture}
\label{sec:new-system-architecture}

The architectural limitations identified in the previous section \ref{sec:architecture-limitations} necessitated a fundamental restructuring of the development approach. The primary challenge was the need to maintain a separate demonstration application (\texttt{em\_demo}) for implementing and testing gamification features while preserving the proprietary nature of the main English Mind application. This requirement, combined with the existing architectural constraints of component coupling and documentation gaps, led to the adoption of a monorepo architecture. This new architecture provides a structured environment that enables parallel development of both applications while maintaining clear separation of concerns. The following sections detail how this architectural shift not only addresses the immediate need for separate applications but also improves overall maintainability and scalability of the development process.

\subsection{Monorepo Solution}

To address the architectural limitations identified in Section \ref{sec:architecture-limitations}, we implemented a monorepo architecture that enables parallel development of both the main English Mind application and the demonstration application. This approach provides several key benefits:

\begin{itemize}
    \item \textbf{Code Reusability} - Shared components and utilities can be extracted into separate packages, reducing code duplication and ensuring consistency across applications.
    \item \textbf{Unified Version Control} - All code resides in a single repository, providing a complete history of changes across all applications and packages, while maintaining clear separation between the main and demonstration applications.
    \item \textbf{Streamlined Development Workflow} - Developers can work on multiple applications simultaneously while maintaining a single source of truth for shared code.
    \item \textbf{Improved Testing and Validation} - New features can be developed and tested in the demonstration application before being integrated into the main application.
\end{itemize}

The monorepo solution was implemented using Melos \cite{cite:melos}, a tool specifically designed for managing Dart and Flutter monorepos. Melos provides essential features such as workspace management, package linking, and script execution across multiple packages. This choice was made to leverage Melos's robust support for Flutter development and its ability to handle complex dependency graphs while maintaining optimal build performance.

\subsection{Monorepo Architecture}

The monorepo architecture is organized into a clear directory structure that promotes code organization, reusability, and maintainability. The root directory contains the following key components:

\begin{itemize}
    \item \textbf{\texttt{apps/}} – Contains all Flutter applications in the workspace.
    \begin{itemize}
        \item \textbf{\texttt{english\_mind/}} – The production version of English Mind.
        \item \textbf{\texttt{em\_demo/}} – The demonstration version of English Mind.
        \item \textbf{\texttt{widgetbook/}} – UI components documentation.
    \end{itemize}

    \item \textbf{\texttt{packages/}} – Houses all shared packages used across applications.
    \begin{itemize}
        \item \textbf{\texttt{em\_theme/}} – Shared UI theming and styling components.
        \item \textbf{\texttt{em\_lints/}} – Custom linting rules and analysis options.
        \item \textbf{\texttt{em\_widgets/}} – Reusable UI components.
        \item \textbf{\texttt{em\_vocabulary/}} – Core vocabulary data, models and management.
    \end{itemize}

    \item \textbf{\texttt{melos.yaml}} – Melos configuration scripts.
\end{itemize}

The \texttt{em\_demo} application follows the same architectural structure as the main English Mind application as described in Section \ref{sec:architectural-overview}, maintaining consistency in code organization and development patterns. This approach ensures that features developed in the demonstration application can be seamlessly integrated into the main application when ready for production deployment.

\subsection{Widgetbook Documentation}

Widgetbook \cite{cite:widgetbook} serves as a dedicated documentation environment for UI components, providing an interactive catalog of all shared widgets and their variations, as shown in Figure \ref{fig:widgetbook}. This tool enables developers to visualize and test components in isolation, ensuring consistent implementation across both applications. 

The adoption of Widgetbook was driven by its ability to provide centralized, interactive documentation of UI components, making it easier for developers to understand and use shared components correctly. The tool significantly improves development efficiency by allowing developers to work on components in isolation without needing to navigate through the full application, reducing context switching and development time \cite{cite:widgetbook_efficiency}. Furthermore, it serves as a living style guide that validates the implementation of the design system and ensures adherence to design specifications, while also facilitating better communication between designers and developers through a shared reference point for UI components \cite{cite:design_system_validation}.

\section{Features Implementation}
\label{sec:features-implementation}

All gamification features proposed in Chapter \ref{chap:proposed-solution-design} were successfully implemented in the demonstration application, as shown in Appendix \ref{app:implementation}. The implementation followed the architectural principles established in Section \ref{sec:new-system-architecture}, ensuring maintainability and scalability. While all features were implemented, this section focuses on the two most significant additions: the diversified flashcard system and the streak system. Detailed implementation specifics can be found in the source code.

\subsection{Diversified Flashcard Types}

The implementation of diversified flashcard types showcases a clean, type-safe approach using Dart's sealed classes. This design pattern ensures exhaustive handling of all flashcard types while maintaining a clear separation of concerns. The core of this implementation is the \texttt{FlashcardData} sealed class hierarchy, see Code \ref{lst:flashcard-data}.

\begin{lstlisting}[language=Dart, caption={FlashcardData sealed class hierarchy defining different flashcard types}, label=lst:flashcard-data]
sealed class FlashcardData {}

class RecallFlashcardData extends FlashcardData {
  final VocabularyEntry entry;
  final FlashcardSrsMetadata metadata;

  RecallFlashcardData({required this.entry, required this.metadata});
}

class SpellingFlashcardData extends FlashcardData {
  final VocabularyEntry entry;

  SpellingFlashcardData(this.entry);
}

class PronunciationFlashcardData extends FlashcardData {
  final VocabularyEntry entry;

  PronunciationFlashcardData(this.entry);
}

class MatchDefinitionsFlashcardData extends FlashcardData {
  final List<VocabularyEntry> entries;

  MatchDefinitionsFlashcardData(this.entries);
}

class MatchTranslationsFlashcardData extends FlashcardData {
  final List<VocabularyEntry> entries;

  MatchTranslationsFlashcardData(this.entries);
}
\end{lstlisting}

This sealed class hierarchy provides several key benefits. The use of sealed classes ensures type safety through exhaustive pattern matching, which guarantees that all possible cases are handled at compile time. This design pattern also promotes a clear separation of concerns between different flashcard types, making the code more maintainable and easier to understand. Furthermore, the architecture is designed with extensibility in mind, allowing for straightforward addition of new flashcard types in the future while maintaining the existing type safety guarantees.

The type-safe design significantly simplifies flashcard queue creation and management. Since each flashcard type is explicitly defined in the sealed class hierarchy, the system can easily generate and manage queues containing any combination of flashcard types. The compiler ensures proper handling of all flashcard types, preventing accidental omissions during queue creation and processing. This design also enables efficient queue operations like filtering, shuffling, and type-specific transformations while maintaining type safety.

The UI implementation leverages this type-safe design through Dart's pattern matching capabilities. The \texttt{StudyScreen} widget demonstrates how this architecture enables clean, maintainable UI code, see Code \ref{lst:study-screen}.

\begin{lstlisting}[language=Dart, caption={StudyScreen widget with pattern matching}, label=lst:study-screen]
@RoutePage()
class StudyScreen extends HookWidget {
  @override
  Widget build(BuildContext context) {
    // ... state management code ...

    return Scaffold(
      body: switch (flashcardData) {
        final SpellingFlashcardData data =>
          SpellingFlashcard(entry: data.entry),
        final MatchDefinitionsFlashcardData data =>
          MatchDefinitionsFlashcard(entries: data.entries),
        final MatchTranslationsFlashcardData data =>
          MatchTranslationsFlashcard(entries: data.entries),
        final PronunciationFlashcardData data =>
          PronunciationFlashcard(vocabularyEntry: data.entry),
        final RecallFlashcardData data => RecallFlashcard(
            vocabularyEntry: data.entry,
            flashcardSrsMetadata: data.metadata,
          ),
      },
    );
  }
}
\end{lstlisting}

This implementation demonstrates several key architectural principles. Pattern matching ensures comprehensive handling of all flashcard types, while dedicated widgets for each type promote the single responsibility principle. The sealed class hierarchy enforces a strict contract where new flashcard types require corresponding UI updates, preventing inconsistencies. The architecture also maintains flexibility for future expansion, allowing straightforward addition of new flashcard types.

\subsection{Streak System}

The streak system implementation focused on creating a reliable and engaging mechanism for tracking daily practice, with a design that prioritizes scalability and future extensibility. The core of the streak system is implemented in the \texttt{Streak} class, see Code \ref{lst:streak-class}.

\begin{lstlisting}[language=Dart, caption={Streak class implementation}, label=lst:streak-class]
class Streak {
  final String _uid;
  final Set<int> _activeDays; // Unix timestamps
  // final Set<int> freezeDays; // Unix timestamps

  bool get isActive;
  int get currentStreak;
  int get longestStreak;

  Streak({required String uid, Set<int>? activeDays})
      : _uid = uid,
        _activeDays = activeDays ?? <int>{};
}
\end{lstlisting}

The streak system's implementation provides a solid foundation for tracking user engagement while maintaining flexibility for future enhancements. The design decisions ensure that the system remains reliable, accurate, and scalable as the application evolves. The implementation demonstrates several key design decisions:

\begin{itemize}
    \item \textbf{Timezone Independence} - By using Unix timestamps (milliseconds since epoch) and converting to local dates only when necessary, the system maintains consistency across different timezones. This approach ensures that streak calculations remain accurate regardless of the user's location or timezone changes.
    
    \item \textbf{Historical Data Preservation} - The system maintains a complete history of active days, enabling not only current streak tracking but also historical analysis. This data structure allows for easy calculation of both current and longest streaks, providing valuable insights into user engagement patterns its analysis.
    
    \item \textbf{Data Integrity} - Streak values are calculated on-demand rather than stored, ensuring that the data remains consistent even if the system rules change. This approach prevents data corruption and allows for flexible updates to streak calculation logic without requiring data migration.
    
    \item \textbf{Future Extensibility} - The commented \texttt{freezeDays} field (line 4 in Code \ref{lst:streak-class}) demonstrates how the system can be extended to support additional features like streak freezes. This design pattern allows for straightforward addition of new functionality without disrupting existing streak calculations.
\end{itemize}